% Options for packages loaded elsewhere
\PassOptionsToPackage{unicode}{hyperref}
\PassOptionsToPackage{hyphens}{url}
%
\documentclass[
]{article}
\usepackage{amsmath,amssymb}
\usepackage{iftex}
\ifPDFTeX
  \usepackage[T1]{fontenc}
  \usepackage[utf8]{inputenc}
  \usepackage{textcomp} % provide euro and other symbols
\else % if luatex or xetex
  \usepackage{unicode-math} % this also loads fontspec
  \defaultfontfeatures{Scale=MatchLowercase}
  \defaultfontfeatures[\rmfamily]{Ligatures=TeX,Scale=1}
\fi
\usepackage{lmodern}
\ifPDFTeX\else
  % xetex/luatex font selection
\fi
% Use upquote if available, for straight quotes in verbatim environments
\IfFileExists{upquote.sty}{\usepackage{upquote}}{}
\IfFileExists{microtype.sty}{% use microtype if available
  \usepackage[]{microtype}
  \UseMicrotypeSet[protrusion]{basicmath} % disable protrusion for tt fonts
}{}
\makeatletter
\@ifundefined{KOMAClassName}{% if non-KOMA class
  \IfFileExists{parskip.sty}{%
    \usepackage{parskip}
  }{% else
    \setlength{\parindent}{0pt}
    \setlength{\parskip}{6pt plus 2pt minus 1pt}}
}{% if KOMA class
  \KOMAoptions{parskip=half}}
\makeatother
\usepackage{xcolor}
\usepackage[margin=1in]{geometry}
\usepackage{graphicx}
\makeatletter
\def\maxwidth{\ifdim\Gin@nat@width>\linewidth\linewidth\else\Gin@nat@width\fi}
\def\maxheight{\ifdim\Gin@nat@height>\textheight\textheight\else\Gin@nat@height\fi}
\makeatother
% Scale images if necessary, so that they will not overflow the page
% margins by default, and it is still possible to overwrite the defaults
% using explicit options in \includegraphics[width, height, ...]{}
\setkeys{Gin}{width=\maxwidth,height=\maxheight,keepaspectratio}
% Set default figure placement to htbp
\makeatletter
\def\fps@figure{htbp}
\makeatother
\setlength{\emergencystretch}{3em} % prevent overfull lines
\providecommand{\tightlist}{%
  \setlength{\itemsep}{0pt}\setlength{\parskip}{0pt}}
\setcounter{secnumdepth}{-\maxdimen} % remove section numbering
% definitions for citeproc citations
\NewDocumentCommand\citeproctext{}{}
\NewDocumentCommand\citeproc{mm}{%
  \begingroup\def\citeproctext{#2}\cite{#1}\endgroup}
\makeatletter
 % allow citations to break across lines
 \let\@cite@ofmt\@firstofone
 % avoid brackets around text for \cite:
 \def\@biblabel#1{}
 \def\@cite#1#2{{#1\if@tempswa , #2\fi}}
\makeatother
\newlength{\cslhangindent}
\setlength{\cslhangindent}{1.5em}
\newlength{\csllabelwidth}
\setlength{\csllabelwidth}{3em}
\newenvironment{CSLReferences}[2] % #1 hanging-indent, #2 entry-spacing
 {\begin{list}{}{%
  \setlength{\itemindent}{0pt}
  \setlength{\leftmargin}{0pt}
  \setlength{\parsep}{0pt}
  % turn on hanging indent if param 1 is 1
  \ifodd #1
   \setlength{\leftmargin}{\cslhangindent}
   \setlength{\itemindent}{-1\cslhangindent}
  \fi
  % set entry spacing
  \setlength{\itemsep}{#2\baselineskip}}}
 {\end{list}}
\usepackage{calc}
\newcommand{\CSLBlock}[1]{\hfill\break\parbox[t]{\linewidth}{\strut\ignorespaces#1\strut}}
\newcommand{\CSLLeftMargin}[1]{\parbox[t]{\csllabelwidth}{\strut#1\strut}}
\newcommand{\CSLRightInline}[1]{\parbox[t]{\linewidth - \csllabelwidth}{\strut#1\strut}}
\newcommand{\CSLIndent}[1]{\hspace{\cslhangindent}#1}
\ifLuaTeX
  \usepackage{selnolig}  % disable illegal ligatures
\fi
\usepackage{bookmark}
\IfFileExists{xurl.sty}{\usepackage{xurl}}{} % add URL line breaks if available
\urlstyle{same}
\hypersetup{
  hidelinks,
  pdfcreator={LaTeX via pandoc}}

\title{\vspace{-2cm}
\begin{flushright}
{\small \textit{Student ID: 7038293}}
\end{flushright}

Analysing Roadkill Trends in the UK: Identifying Spatial and Temporal
Patterns}
\author{}
\date{\vspace{-2.5em}}

\begin{document}
\maketitle

\subsection{Introduction}\label{introduction}

\textbf{Topics:} Data Wrangling, Advanced Visualisation, RMarkdown,
Hierarchical Modelling, Spatial Visualisation, Temporal Trends

With the ever-changing climate, British wildlife face an uncertain
future, with a major decline in biodiversity being found all around the
United Kingdom. But another thing that could be adding to the decline in
British wildlife is road-traffic-accidents; collisions with vehicles is
one of the major causes of wild animal death in the UK {[}1{]}. For
animals that are already under immense pressure due to anthropogenic and
climate threats, roadkill could push some of these species to
extinction.

This report aims to provide a comprehensive exploration of roadkill
trends across the UK, leveraging data science techniques to address
questions about its spatial, temporal and environmental dynamics, with a
specific focus on mammals.

Focusing on mammals is particularly relevant due to their ecological
importance and the disproportionate impact of roadkill on their
populations. Many mammal species are wide ranging with large territories
that are increasingly being split up by roads, making them more
vulnerable to collisions. By narrowing the scope to mammals, this study
aims to provide actionable insights for mitigating roadkill impacts on
vulnerable mammalian species.

The four main questions I am hoping to answer are:

\begin{enumerate}
\def\labelenumi{\arabic{enumi}.}
\item
  What are the spatial and seasonal trends in mammalian roadkill
  numbers?
\item
  Do temperature or rainfall play a role in mammalian roadkill
  prevalence?
\item
  Where are the mammalian roadkill hotspots across the UK?
\item
  Which mammalian species are most commonly recorded as roadkill?
\end{enumerate}

\subsection{Objectives/Methodology}\label{objectivesmethodology}

\subsection{Data}\label{data}

\subsubsection{Roadkill Data}\label{roadkill-data}

The data I will use is from The Road Lab (formerly `Project Splatter')
{[}\url{https://www.theroadlab.co.uk/}{]}, it is a citizen science
project with 57 columns and 68,212 rows of data. The dataset includes
mammals, birds, amphibians and reptiles but, as this report is only
interested in mammals, all rows corresponding with other Orders are
removed.

Data was collected between 01/01/2014 until 30/09/2024 but as there was
only one entry and it was not the complete year, data from 2024 was
removed. There were several recordings of ``indet. Deer'' which were
removed along with any other sightings that weren't confirmed to a
species level, so in total \textbf{X} were removed.

Once all unneeded data was removed, I was left with \textbf{X} entries.

\subsubsection{Seasonal/Weather Data}\label{seasonalweather-data}

This data will then be combined with data on the mean monthly
temperature and total rainfall per month from the Met Office.

\subsubsection{Spatial/Traffic Data}\label{spatialtraffic-data}

Mean monthly traffic levels will also be used when looking into roadkill
hotspots as a potential explanation for these hotspots.

\subsection{Results}\label{results}

\subsubsection{Exploratory Analysis}\label{exploratory-analysis}

What I want to include:

\begin{itemize}
\item
  Overall number of reports per month
\item
  Reports per season
\item
  Multiple maps of the UK showing reports per season
\item
  Overall species' barchart
\end{itemize}

\subsubsection{Seasonal trends in
roadkill}\label{seasonal-trends-in-roadkill}

\subsubsection{Spatial Trends in
roadkill}\label{spatial-trends-in-roadkill}

\subsection{Limitations}\label{limitations}

\begin{itemize}
\item
  Larger species are more likely to be seen and reported
\item
  Only covers animals that die immediately and not those that get out of
  the road before dying
\item
  Citizen science always comes with limitations~~~~~
\item
  Bias in sampling
\item
  However, previous studies in South Africa (REF:
  \url{https://www.frontiersin.org/journals/ecology-and-evolution/articles/10.3389/fevo.2018.00015/full})
  ~and California have shown the identification data to be reliable
  (REF:
  \url{https://www.frontiersin.org/journals/ecology-and-evolution/articles/10.3389/fevo.2017.00089/full})
\end{itemize}

\subsection{Conclusion}\label{conclusion}

\subsection{Reproducibility}\label{reproducibility}

All code and data files used for the analysis in this report can be
found in a GitHub repository here:

\subsection*{References}\label{references}
\addcontentsline{toc}{subsection}{References}

\phantomsection\label{refs}
\begin{CSLReferences}{0}{0}
\bibitem[\citeproctext]{ref-raymond_temporal_2021}
\CSLLeftMargin{{[}1{]} }%
\CSLRightInline{S. Raymond, A. L. W. Schwartz, R. J. Thomas, E.
Chadwick, and S. E. Perkins, {``Temporal patterns of wildlife roadkill
in the {UK},''} \emph{PLOS ONE}, vol. 16, no. 10, p. e0258083, Oct.
2021, doi:
\href{https://doi.org/10.1371/journal.pone.0258083}{10.1371/journal.pone.0258083}.}

\end{CSLReferences}

\end{document}
